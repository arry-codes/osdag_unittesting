\documentclass[a4paper,12pt]{article}
\usepackage[utf8]{inputenc}
\usepackage{geometry}
\usepackage{fancyhdr}
\usepackage{graphicx}
\usepackage{hyperref}
\usepackage{listings}
\usepackage{xcolor}
\usepackage{longtable}
\usepackage{titlesec}
\usepackage{booktabs}

\titleformat{\section}
  {\normalfont\Large\bfseries}
  {(\thesection)}
  {0.5em}
  {}

\titleformat{\subsection}
  {\normalfont\large\bfseries}
  {(\thesubsection)}
  {0.5em}
  {}

\titlespacing*{\subsection}
  {20pt}
  {3.25ex plus 1ex minus .2ex}
  {1.5ex plus .2ex}

\geometry{
 a4paper,
 total={170mm,257mm},
 left=20mm,
 top=20mm,
}

\definecolor{codegreen}{rgb}{0,0.6,0}
\definecolor{codegray}{rgb}{0.5,0.5,0.5}
\definecolor{codepurple}{rgb}{0.58,0,0.82}
\definecolor{backcolour}{rgb}{0.95,0.95,0.92}

\lstdefinestyle{mystyle}{
    backgroundcolor=\color{backcolour},   
    commentstyle=\color{codegreen},
    keywordstyle=\color{magenta},
    numberstyle=\tiny\color{codegray},
    stringstyle=\color{codepurple},
    basicstyle=\ttfamily\footnotesize,
    breakatwhitespace=false,         
    breaklines=true,                 
    captionpos=b,                    
    keepspaces=true,                 
    numbers=left,                    
    numbersep=5pt,                  
    showspaces=false,                
    showstringspaces=false,
    showtabs=false,                  
    tabsize=2
}
\lstset{style=mystyle}

\begin{document}

\begin{center}
    \vspace*{-0.5cm}
    {\LARGE \underline{\textbf{Osdag Unit Testing Project Report}}} \\
    \vspace{0.5em}
\end{center}
\thispagestyle{plain}

\section{Introduction}

This report documents the unit testing framework developed for the Osdag project. Osdag is an open-source software for the design of steel structures. The objective of this project was to verify the accuracy of the Osdag computational modules by comparing their outputs against known validated results.

\section{Methodology}

\subsection{Test Framework}
\textbf{PyTest} was selected as the testing framework due to its flexibility and powerful assertion introspection.

\subsection{Data Sources}
Input parameters and expected output values were sourced from CSV files provided for specific modules:
\begin{itemize}
    \item \texttt{CleatAngle.csv}
    \item \texttt{FinPlate.csv}
    \item \texttt{TensionMember.csv}
\end{itemize}

\subsection{Parameter Mapping}
A key challenge was mapping the headers in the CSV files to the internal dictionary keys returned by the Osdag modules. A mapping dictionary \texttt{KEY\_MAPPING} was implemented in \texttt{test\_osdag.py} to bridge these naming conventions.

\section{Measured Modules}

The unit tests cover three primary connection/member types:

\begin{enumerate}
    \item \textbf{Cleat Angle Connection}
    \item \textbf{Fin Plate Connection}
    \item \textbf{Tension Member}
\end{enumerate}

\section{Results}

The unit tests were executed to validate the module outputs against the reference CSV data.

\subsection{Test Summary}
\begin{itemize}
    \item \textbf{Total Tests Collected}: 48
    \item \textbf{Tests Passed}: 48
    \item \textbf{Tests Skipped}: 0
    \item \textbf{Tests Failed}: 0
\end{itemize}

\subsection{Execution Log}
The following describes two stages of testing verification. 

\subsubsection{Initial Run (File-based)}

\begin{lstlisting}[language=bash, caption=Initial PyTest Execution Output]
============ test session starts ============
platform darwin -- Python 3.13.0 , pytest -9.0.2 , pluggy -1.6.0
rootdir : / Users / aryan / Desktop / Unit Testing
plugins : anyio -4.12.0
collected 12 items

test_osdag . py ............ [100%]

============ 12 passed in 2.05 s =============
\end{lstlisting}

\subsubsection{Detailed Run (Granular)}

\begin{lstlisting}[language=bash, caption=Granular PyTest Execution Output]
============ test session starts ============
platform darwin -- Python 3.13.0, pytest-9.0.2, pluggy-1.6.0
rootdir: /Users/aryan/Desktop/Unit Testing
collected 48 items

test_osdag.py::TestOsdagModules::test_designation[CleatAngleTest1] PASSED
test_osdag.py::TestOsdagModules::test_shear_capacity[CleatAngleTest1] PASSED
test_osdag.py::TestOsdagModules::test_bolt_configuration[CleatAngleTest1] PASSED
...
test_osdag.py::TestOsdagModules::test_capacity[TensionWeldedTest4] PASSED

======= 48 passed in 2.90s =======
\end{lstlisting}

\section{Conclusion}

The unit testing suite has successfully verified the core functionality of the selected Osdag modules. All 12 test cases matched the expected values defined in the CSV files within the specified tolerances.

\end{document}
